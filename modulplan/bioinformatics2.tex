\module{compulsory}
{lecture and excercise}
{German}
{9}
{270 h}
{The lecture focuses on the fundamental algorithms of bioinformatics. In the accompanying exercises, the student should gain practical experience in the application of standard tools of bioinformatics to life sciences questions, but on the other hand, the writing of own computer programs should be practiced. Great emphasis is placed on deepening the acquired knowledge in accompanying exercises in small groups. This compulsory module is the basis of all other bioinformatics events. Contents of the lecture are: Pairwise alignments, multiple alignments, BLAST, phylogeny, Markov models, machine learning, RNA secondary structure, protein secondary structure, protein-tertiary structure, microarrays, sequencing.
} 
{The students know basic concepts and methods of bioinformatics as well as mathematical methods for the modeling of biological problems. By dealing with typical bioinformatic questions, the students are prepared to cope with the situations that arise in their daily work. They can recognize biological problems and describe, abstract and then solve them as bioinformatic problems.
}
{None}