\module{compulsory}
{lecture and laboratory excercise}
{German}
{12}
{360h}
{\paragraph{Lecture Biomolecules and Cell:} The lecture gives a brief outline of the biochemical basis of life, introduces the basic structures of eukaryotic and prokaryotic cells and describes the principles of cell growth and proliferation. It explains the molecular basis of genetic information, the flow of genetic information from DNA to protein and the consequences of mutation and recombination. In addition to an insight into the basics of bacteria and viral genetics, an introduction to genetic engineering is given.
\paragraph{Laboratory Excercise Biomolecules and Cell:} The practical part of Biomolecules and Cell deals with the following topics:
Construction and detection of DNA and proteins;
microscopy; basics of cell biology; layout and organells of eukaryotic cells; introduction to genetics.
\paragraph{Lecture Molecular Biology I (Cell biology and Genetics): } The lecture presents the molecular mechanisms of cell proliferation, cell death, and cell motility, and addresses the more complex functions of cells for metabolism, differentiation, signaling, and development. It deals with the organization of genes in the genome, selected mechanisms of gene regulation and the principles of developmental genetics. It explains the methods used by molecular cell biology and molecular genetics.}
{The students can observe and reproduce biological phenomena in detail, identify and describe organisms, create scientific records, analyze and interpret measurement and test results, adequately select subject-specific techniques, document measurement and test results, and communicate biological issues. They have proven to understand scientific contexts, work critically and develop a sound professional judgment. During the laboratory excercise they have worked in a team.}
{None}