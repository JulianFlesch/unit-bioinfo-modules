\chapter*{Preface}

\section*{Structure and Contents}

This list describes all compulsory and a possible selection of compulsory elective units required for the Bachelor program Bioinformatics at the Eberhard Karls University Tübingen.
A unit can be either a lecture, laboratory work, seminar, project, or a thesis.

\section*{Credits Points}

The individual units are each assigned points complying with the European Credit Transfer System (ETCS). Credits are a quantitative measure of the time that has to be spend to complete a unit. 
They are coupled to the successful participation in courses and generation of academic achievments.
Credits cover the actual teaching time in the courses (attendance time) as well as the time for the preparation and follow-up of the course material (self-study), the effort for the individual achievements (study achievements and exam preparation and for the bachelor thesis) as well as for internships.
 
Typically, 60 credits are awarded per academic year; 30 credits per semester. 
One ETCS point is earned for an effort of 30 hours in self-study, classrooms, or labs.
The total workload should not exceed 900 hours in the semester, including the lecture-free period, or 1.800 hours in the academic year.
This sums up to 45 weeks of 40 hours of work.

\section*{Types of Units}

\paragraph{Lectures} are (if not described in detail) a series of events in which the transfer of knowledge takes place by means of lectures by the lecturer. Lectures are often accompanied by exercises in which the topics of the lecture are applied, deepened or repeated. Often there are event-related exercise sheets. In addition, there are presence or programming exercises in many events, in which subjects suitable for the lecture are dealt with under direct supervision. Grading usually results from the result of an exam (or oral exam) at the end of the lecture.

\paragraph{Seminars} Seminars are (if not stated differently) a series of events in which students get familiar with an assigned topic and prepare a lecture to the lecturer and fellow students. 
As a rule, a written elaboration is also required. 
Coursework and examinations are typically provided in the form of a lecture, a written preparation and active participation in the discussions. 

\paragraph{Laboratory or project work}
Laboratories or projects are (if not stated differently) events in which students work on an assignments independently or under supervision independently or in small groups.
Students are usually evaluated based on their participation, the presentation of results, and written reports.

\section*{Grading}
Each unit is typically completed with a grade determined by having a single exam.
In exceptional cases, the grades may also be based on multiple tests. 
The evaluation is carried out by the lecturers of the respective unit. 
In accordance with the examination regulations, the module grades are weighted with their credit points into the final grade (Bachelor certificat grade).

