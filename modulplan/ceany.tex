\module{compulsory elective}
{lecture and excercise}
{German/English}
{6}
{180 h}
{The module provides advanced knowledge of bioinformatics, informatics or life sciences. These are acquired in selected lectures -- e.g:
\paragraph{Lecture Microarray Bioinformatics:} Fundamentals of the technologies for the expression analysis esp. Of Microarrays, algorithms for the
quantification of expression data, basic statistical methods for calculating differential
expression and machine learning (including clustering and classification) for. These are statistical methods, such as those in the stochastic lecture
be applied to specific questions, especially real microarray experiments in this area.
In the accompanying exercises, the content of the lecture will be deepened with blended and peer learning methods.
Students learn and practice programming with R.}
{The students have further knowledge in bioinformatics, in informatics or in the life sciences and can apply these. The qualifications obtained in the corresponding courses for this module enable the students to familiarize themselves with a field of bioinformatics, informatics or life sciences in order to develop their professional knowledge. Depending on the event, for example, students acquire in-depth skills in practical laboratory work or in-depth knowledge of bioinformatics, informatics or life science topics. The student is thus given the opportunity to set up a profile with a focus on one of the three mentioned areas.
}
{None}