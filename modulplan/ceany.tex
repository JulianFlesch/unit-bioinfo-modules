\module{compulsory elective}
{lecture and excercise}
{German/English}
{6}
{180 h}
{The module provides advanced knowledge of bioinformatics, informatics or life sciences. These are acquired in selected lectures -- e.g:
\paragraph{Lecture Microarray Bioinformatics:} Fundamentals of the technologies for the expression analysis esp. of Microarrays. 
Topics are among others: Algorithms for microarray design, image analysis, normalization methods, dimension reduction by principal component analysis, clustering, statistical hypothesis testing, and classification techniques.}
{Methods and acquired skills of the various modules of the first two years of study (for example, algorithms, statistical methods, programming skills) are applied to concrete questions of an important topic of bioinformatics. 
The students analyze microarray experiments and learn how to program with the scripting language R.
They understand the connections between the different aspects of what has been learned so far and can apply them to practical problems.
They are able to actively capture problems, discuss them critically and develop solutions. This increases the methodological competence of the student.}
{Bioinformatics II and Stochastics}