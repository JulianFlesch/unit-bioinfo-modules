\module{compulsory}
{lecture and excercise}
{German}
{3}
{90 h}
{\paragraph{Lecture Bioinformatics I (Introduction to Bioinformatics): } 
The module provides an insight into bioinformatics. For this purpose, selected topics of bioinformatics will be briefly presented in the lecture. It will be put into perspective, how the modules from the first two years of bioinformatics will prove useful. The topics are presented by different lecturers at the university of tübingen to cover the entire breadth of bioinformatics. The topics are reviewed and summarized by the students in exercises. A selection of recurring topics is: 
\begin{enumerate}
\item What is bioinformatics? 
\item From DNA to the database: sequencing, assembly, 
\item Darwin's heirs: Genome-based genealogies
\item Metagenomics - from a handful of earth
\item Molecular Machines - Protein structures and their function 
\item Designer Drugs - active ingredients from the computer
\item Vaccinations against cancer - bioinformatics in vaccine design 
\item Analysis of biological networks
\item It's hip to chip - from microarrays to personalized medicine
\item The language of proteins - evolution of conserved protein structures.
\end{enumerate}} 
{The students have gained an overview of the major branches of bioinformatics, such as sequencing, phylogeny, metagenomics and drug design, know the how the subdomains intertwine, and can classify them. The interest of the students in the basic events will be strengthened and the motivation for the professional breadth of the study will be conveyed.}
{None}