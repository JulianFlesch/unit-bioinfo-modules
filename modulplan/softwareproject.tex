\module{compulsory}
{lecture and project work in small teams}
{German}
{9}
{270 h}
{\paragraph{Lecture Software Project: }The module deals with the topics Introduction to Software Engineering, Programming in general, Project organization, Module concept, design by contract, Requirements specification. Specifications, design patterns (observer, model-view-controller, adapters, proxy), events and news, code reviews, unit tests and project documentation.
\paragraph{Project Work Software Project: } The main part of this module is the project work. The students work in small teams on a programming project over the course of one semester. Every group is intensively supervised by a tutor, who gives feedback on the group's progress and helps to sort out issues. The identification of sub-tasks, scheduling, and implementation work is carried out by the group independently. After the semester ends, each group is graded based on the success and presentation of their project.} 
{Students know methods and techniques for the design and programming of complex software in a team and can use them professionally. They can present their own contributions to the overall project clearly and competently and react flexibly to necessary changes. In addition, they can independently organize their project and determine the progress of the project. Students have acquired vocationally oriented, interdisciplinary skills. This may include, but is not limited to, presenting, organizing, communicating, problem-solving, and critically questioning.}
{Computer Science I and II}