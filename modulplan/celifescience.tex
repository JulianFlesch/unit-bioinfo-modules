\module{compulsory elective}
{lecture and seminar}
{German/English}
{6}
{180 h} 
{\paragraph{Lecture Introduction to Immunology} (3 ECTS) Topics are the basics of immunology: Involved cells, development and differentiation, effectors, information transfer, infection control, molecular recognition mechanisms. It gives an overview of the most important cell populations of the immune system, effector functions, plasticity and differentiation processes, overview of immunologically relevant molecular interactions.
\paragraph{Seminar Milestones of Immunology} (3 ECTS) The aim of the seminar is to record, discuss and contextualise findings of immunology and their influence on today's research and medicine.
In the first seminar lesson students will select an article from a topic overview for their presentation and discuss their structure.
Then the students present and discuss 1-2 articles weekly in the seminar (presentation 30 min and subsequent discussion 10 min).
The discussion round will be conducted alternately by a seminar participant. 
The aim of the debriefing is to give the student feedback on his presentation and to question the article critically and constructively.
Presentation in German or English.}
{The students have advanced their knowledge in a field of life sciences and can apply it.
They have successfully presented and presented an immunological publication.}
{Animal Physiology (Neurobiology), Biomolecules and Cell, Chemistry I, Physical Chemistry (Chemistry II)}