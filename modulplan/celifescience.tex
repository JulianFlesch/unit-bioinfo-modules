\module{compulsory elective}
{lecture and excercise}
{German}
{6}
{180 h}
{The module provides advanced knowledge in a selected field of life sciences (biology, chemistry, pharmacy). These are acquired in selected events of biology, chemistry or pharmacy. These selected events include e.g. \textbf{Molecular Biology 2} (6 ECTS total), which deals with two subfields: microbiology (introduction to general microbiology, prokaryotic microbiology, bacterial cell structure and structure, genetics and regulation, metabolism, taxonomic-systematic overview, important bacterial groups and their ecological, economic or medical importance) and plant physiology (molecular plant physiology, aspects of transport physiology and nutrient uptake, physiology of nutrient assimilation and hormonal action, photosynthesis and molecular biology of photomorphogenetic effects of light, biochemistry of phytochemicals and their function, stress physiology). Further courses are \textbf{Introduction to Immunology} with a lecture (3 ECTS) and an accompanying seminar (3 ECTS), as well as the research course \textbf{Computational Methods in Drug Discovery} (6 ECTS). Other courses are also eligible for this elective subject area.}
{The students have advanced knowledge of the life sciences and can apply them. The qualifications gained in the corresponding courses for this module (see corresponding description of the event) allow the students to familiarize themselves with a field of life sciences in order to develop their professional knowledge. Depending on the event, students acquire, for example, in-depth skills in practical laboratory work or in-depth knowledge of life sciences topics.}
{Animal Physiology (Neurobiology), Biomolecules and Cell, Chemistry I, Physical Chemistry (Chemistry II)}